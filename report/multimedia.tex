\documentclass[main.tex]{subfiles}

\begin{document}

\section{电影海报检索}

\subsection{应用场景}

本来我们想实现利用中文海报搜索对应电影信息的功能,但后来想到对于中国人来说,直接从海报中读出标题后在网上搜索更为方便快捷,不存在对于这个功能的需求,所以这个功能改为利用英文海报搜索对应的中文电影信息,受众是不太熟悉英文的人群。

\subsection{思路}

实现这个功能最关键的障碍在于我们无法使用各种快速哈希算法,因为我们希望这个功能的使用场景是用户利用摄像头拍摄海报照片上传至网页进行搜索,而大多数的图片哈希算法或者图片签名算法如 \cite{wong2002image} 要求被搜索图片在数据集中。同样,课程中讲到的 LSH 算法也不大满足我们的需求,因为因为用户上传的照片中有与海报无关的场景信息,这些场景信息对于 LSH 的计算来说是很大的干扰,所以最终决定利用 OCR 技术。

\subsection{实现}

以下为利用英文海报图片搜索电影信息的流程:

首先,我们利用 EAST 文本检测器 \cite{zhou2017east} 检测电影海报中的文字区域。比起直接对整张海报进行 OCR,先检测文字区域在对文字区域进行 OCR 的准确率更高,速度更快。EAST 检测文字的流程如下:

\begin{enumerate}
    \item 先用一个通用的网络(论文中采用的是 Pvanet,实际在使用的时候可以采用 VGG16,Resnet等)作为 base net,用于特征提取;
    \item 基于上述主干特征提取网络,抽取不同 level 的 feature map(它们的尺寸分别是 inuput-image 的 1/32,1/16,1/8,1/4),这样可以得到不同尺度的特征图。目的是解决文本行尺度变换剧烈的问题,early stage 可用于预测小的文本行,late stage 可用于预测大的文本行;
    \item 特征合并层,将抽取的特征进行 merge。这里合并的规则采用了 U-net 的方法,合并规则:从特征提取网络的顶部特征按照相应的规则向下进行合并,具体参见图~\ref{fig:east-net} 中的网络结构图;
    \item 网络输出层,包含文本得分和文本形状。根据不同文本形状可分为 RBOX 和 QUAD),输出也各不相同,具体参看图~\ref{fig:east-net}。
\end{enumerate}

\begin{figure}[h]
    \centering
    \includegraphics[width=0.5\linewidth]{images/east_net.png}
    \caption{EAST 文本检测网络结构图 \cite{zhou2017east}}
    \label{fig:east-net}
\end{figure}

EAST 的最大特点是能够检测倾斜的文字,所以下一步我们需要通过旋转矫正倾角较大的文字区域。然后,计算每一个文字区域的高度并按照文字高度排序,保留文字高度靠前的文字区域(我们的实现中选择的是文字高度前五的文字区域)。从原图中裁剪下这些文字区域进行 OCR。

OCR 方面我们利用的是 Google 的开源软件 Tesseract。Tesseract 的 OCR 模式选择单行文本识别和基于其内置神经网络的 OCR 算法。我们将每一个文字区域的文字识别结果收集起来进行文本搜索。

\subsection{评估}

图~\ref{fig:poster-good-for-ocr} 中给出了一些识别效果较好的电影海报,而图~\ref{fig:poster-bad-for-ocr} 中给出了一些具有代表性的识别效果较差的海报。从这些海报中我们大致可以总结出识别效果差的原因:

\begin{enumerate}
    \item 识别效果差很大程度上是 OCR 的原因而不是文字检测的原因;
    \item OCR 提高准确率的关键在于字体训练,而海报标题中有很多风格迥异的花体或艺术体的字母,Tesseract 就无法识别。例如,图~\ref{fig:poster-bad-for-ocr} 中的第一张和第三章海报无法识别出标题很可能仅仅是因为海报标题的字体中笔画之间有间隙。
\end{enumerate}

\begin{figure}
    \centering
    \includegraphics[width=0.3\linewidth]{images/poster_good_1.png}
    \includegraphics[width=0.3\linewidth]{images/poster_good_2.png}
    \includegraphics[width=0.3\linewidth]{images/poster_good_3.png}
    \caption{识别效果好的电影海报(淡蓝色框为文字区域)}
    \label{fig:poster-good-for-ocr}
\end{figure}

\begin{figure}
    \centering
    \includegraphics[width=0.3\linewidth]{images/poster_bad_1.png}
    \includegraphics[width=0.3\linewidth]{images/poster_bad_2.png}
    \includegraphics[width=0.3\linewidth]{images/poster_bad_3.png}
    \caption{识别效果差的电影海报(淡蓝色框为文字区域)}
    \label{fig:poster-bad-for-ocr}
\end{figure}

\section{电影搜索}

\subsection{应用场景}

当用户走在街上时,被屏幕中有关最新上映电影的预告片所吸引,而又不想停下脚步等待预告片播完时最终出现的电影名字,就可以拍下预告片并上传到我们的系统进行搜索来获取电影信息。

\subsection{思路}

我们将预告片转换为一系列的帧来进行图像处理,而这样会导致数据集中的图片数量过于巨大,而我们又希望能够在合理的时间内返回结果,所以最终采用的仍然是图片搜索转换成文字搜索的这一主要思路。首先是识别预告片中的字幕,然后利用字幕进行文字搜索,寻找对应的电影。

\subsection{实现}

\end{document}
