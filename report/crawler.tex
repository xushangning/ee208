\documentclass[main.tex]{subfiles}

\begin{document}
\section{爬虫}
\subsection{爬取内容信息介绍}

我们组结合电影搜索引擎在现实生活中的相关功能需要,通过问卷调查以及网上查询等方式,确定了我们搜索引擎的期望提供的信息:
\begin{enumerate}
	\item 电影的名字(中文名字与英文名字)
	\item 导演的名字与人物照片
	\item 演员的名字与人物照片
	\item 电影的评分
	\item 电影的短评
	\item 电影的预告片
	\item 电影的海报(中文版与英文版/从IMDB上爬取)
\end{enumerate}
在这里需要一提的是我们为什么需要爬取预告片以及专门爬取电影的英文海报与英文名字:
\begin{enumerate}[1)]
	\item 爬取的预告片是用于电影搜索过程中实现针对预告片中的某一帧进行搜索这一功能目的
	\item 电影的中英文名并存则是针对中国人在搜索电影时因为无法准确翻译电影英文名而造成的搜索困难,同时也是为国外使用者提供了方便
	\item 专门爬取了英文海报则是为了提高搜索的准确性与快捷性,也是为不太熟悉英文的人群以检索英文海报提供方便
\end{enumerate}
\subsection{爬取过程实现}
我们才开始时是希望将电影海报图片与电影基本信息集中在一个爬虫代码里实现,但发现由于发出的request过于繁杂,以及在爬取过程中遇到的种种问题(如爬取预告片时,出现爬取不完整的情况;爬取短评或是人物图像中无相关图片及短评的情况,我们决定分别利用四个爬虫分别爬取电影的文本信息(中英文名字;导演,演员名字;评分;5个短评),电影的预告片,电影的中文海报,电影的英文海报。\par
我们通过查找相关博客,直接获取了豆瓣网存储电影信息的API接口,故我们的爬虫查询都是基于该端口特点进行设计,同时,对于短评的网站接口也是直接根据URL的构成特点进行简单的字符串合并删减找到,大大提高了爬取速度,也是鉴于此种情况,我们没有设置多线程进行搜索。\par
而针对可能的反爬虫机制,我们并没有采用IP池的方法,而是简单的设置sleep时间,避免了相关可能情况;同时,为了解决爬取过程由于各种原因突然中断的情况,我们实现了断点续爬功能,将爬取的文件进行备份,下次直接从上次爬取的地址开始往下爬。

\end{document}